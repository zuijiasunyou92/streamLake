%!TEX root = ../main.tex
\section{STREAM AND TABLE STORAGE OBJECT} 
\label{sec:datagen}

In this section, we introduce the stream object and table object, purpose-built storage abstractions designed for efficient storage and access of stream and table data in the storage layer.


\subsection{Stream Object}

The stream object is a storage abstraction in the store layer that efficiently supports key-value message streaming at scale. It stores a partition\footnote{-- one partition or partitions} of key-value pairs for continuous message streams, organized as collections\footnote{-- a collection or collections} of data slices. Each slice can contain up to 256 records as depicted in Figure 3. Incoming message records are appended  to a specific slice in a stream object based on its topic, key, and offset.

\noindent \textbf{Stream objects operations.} The stream object operates similarly to the block and file storage abstractions, providing read and write functionality for stream storage. Figure 2 outlines key operations supported by the stream object, including creating and destroying a stream object with  functions \texttt{CreateServerStreamObject} (line 1-3) and \texttt{DestroyServerStreamObject} (line 5-6) respectively. The \texttt{$^*$option} field (line 2) sets storage configurations, such as data redundancy methods (replicate or erasure code) and I/O quotas, so as to ensure enterprise-level reliability and performance. The assigned \texttt{objectId} (line 3) serves as a unique identifier for operating the stream object. The \texttt{AppendServerStreamObject} function appends incoming records\footnote{-- an incoming record?}  to the stream object and returns the starting offset of the appended records. The \texttt{ReadServerStreamObject} function reads the stream object starting from a specified offset, with control conditions such as the length of the read specified in the \texttt{readCtrl} field. 
Since the message service is designed to support real-time streaming, it is configured to return all subsequent messages\footnote{*why?} unless \add{specified or reaching quota limits.}\footnote{*The context of this sentence is not clear.} \texttt{IO\_CONTENT\_S} (line 10 and 17) is a data structure that provides non-blocking I/O by using buffers to enhance the performance of both writing and reading operations.



\noindent \textbf{Write stream messages.} We discuss how to write messages into \sys and endure enterprise-level load-balanced and redundant persistence for the stream objects, which is achieved on the basis of SSD and HDD storage pools.  As shown in Figure 3, the messages are first   assigned to stream object slices based on topics, keys, and offsets (Figure 3-a,b,c).\footnote{@ how dies it work?} Then, a distributed hash table is leveraged to ensure even data distribution for load-balance storage (Figure 3-d). Specifically, data slices will be distributed evenly to 4096 logical shards, each of which has the storage space managed by persistence logs (PLog, Figure 3-e)\footnote{--Plog 4096}. Each PLog unit is a collection of persistence services in OceanStor~\cite{} that controls a fixed amount of storage space on multiple disks and provides 128 MB of addresses per shard. When a message is received, the PLog unit replicates it to multiple disks for redundancy (Figure 3-f). Key-value databases\footnote{--can we name it specifically?} serve as indexes for PLogs for fast record lookup.


\subsection{Table Object}
We also extend the storage object layer in StreamLake to support table-like\footnote{-- a term?} operations for more effective data storage and management, similar to lakehouses~\cite{}\footnote{* ``similar to lakehouse'' is weird}. The table storage uses an open lakehouse format\footnote{--a term? what does it mean?} with optimizations\footnote{--what? a part in Fig4?} for faster metadata access. The table abstraction is logically defined by a directory of data and metadata files, as shown an example  in Figure 4.

\noindent \textbf{\texttt{Data} directory.} Table objects are stored in Parquet files of the \texttt{data} directory. In this example, the table is partitioned based on the date column, so the data objects are separated into different sub-directories by date. Each sub-directory name represents its partition range\footnote{--partition key?}. The data objects in each Parquet file are organized as row-groups\footnote{--a term?} and stored in a columnar format for efficient data analysis. Footers in the Parquet files contain statistics to support data skipping within the file\footnote{--remove within the file?}.

\noindent \textbf{\texttt{Metadata} directory}  keeps track of the table schema, file addresses of the table, its partitions and transaction commits \add{etc.}\footnote{*how to correspond to the three levels? and the following illustration order?}, which are organized into three levels: commit, snapshot, and catalog, as shown in Figure 4-(b, c, d).

 \noindent \underline{\textit{Commits}} are \texttt{Arvo} files that contain file-level\footnote{--a term?} metadata and statistics such as file paths\footnote{@directory path}, record counts, and value ranges for the data objects. Each data insert, update, and delete operation will generate a new commit file to record changes to the data object files\footnote{--data object files or data object?}.


\noindent \underline{\textit{Snapshots}} are index files that index  valid commit files for a specified time period. These snapshots document commit statistics such as \add{current file and row count, added files and rows, and removed files and rows}\footnote{--messy} as data operation logs. Along with commits, snapshots provide snapshot-level isolation to support optimistic concurrency control. Readers can access the data by reading from the valid commit files, while changes made by a writer will not be visible to readers until they are committed and recorded in a snapshot. This allows for multiple readers and one writer to access the data simultaneously without the need for locks. 

Snapshots also monitor the expiration of all commits, making them essential for supporting time travel. Time travel queries allow data to be viewed as it appeared at a specific time\footnote{@}. By keeping old commits and snapshots, the table object enables the use of a timestamp to look up the corresponding snapshot and commits, so as to access to historical data.

\noindent \underline{\textit{Catalog}}  describes the table object, including the profile data  such as the table ID, directory paths, schema, snapshot descriptions, modification timestamps, etc. The data and metadata files are stored in the table directory\footnote{@}, except for the catalog, which is stored in a distributed key-value engine~\footnote{--can we name it? or cite} optimized for RDMA and SCM to ensure fast metadata access. The data and metadata files are converted to PLogs in the underlying storage for redundant persistence as discussed above.






















