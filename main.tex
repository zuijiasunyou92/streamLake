% VLDB template version of 2020-08-03 enhances the ACM template, version 1.7.0:
% https://www.acm.org/publications/proceedings-template
% The ACM Latex guide provides further information about the ACM template

\documentclass[sigconf, nonacm]{acmart}

%% The following content must be adapted for the final version
% paper-specific
\newcommand\vldbdoi{XX.XX/XXX.XX}
\newcommand\vldbpages{XXX-XXX}
% issue-specific
\newcommand\vldbvolume{14}
\newcommand\vldbissue{1}
\newcommand\vldbyear{2020}
% should be fine as it is
\newcommand\vldbauthors{\authors}
\newcommand\vldbtitle{\shorttitle} 
% leave empty if no availability url should be set
\newcommand\vldbavailabilityurl{URL_TO_YOUR_ARTIFACTS}
% whether page numbers should be shown or not, use 'plain' for review versions, 'empty' for camera ready
\newcommand\vldbpagestyle{plain} 


%!TEX root = ../main.tex
\newcommand{\eat}[1]{}
\usepackage{latexsym}
\usepackage{amsfonts}
\usepackage{amsmath}
%\usepackage{amssymb}
\usepackage{color}
\usepackage{colortbl}
\usepackage{epsfig}
\usepackage{xspace}
\usepackage{graphicx}
\usepackage{subfigure}
\usepackage{pifont}
\usepackage{bm}

%\usepackage{algorithmicx}
\usepackage{xparse}

\usepackage[lined,boxed,vlined,ruled,linesnumbered]{algorithm2e}
\usepackage{algorithmicx}
\usepackage{paralist}
\usepackage{enumerate}
\usepackage{ifthen}
\usepackage{ulem}
\usepackage{makecell}

%response command
\newcommand{\stitle}[1]{\vspace{1.2ex}\noindent{\bf #1}}
\newcommand{\etitle}[1]{\vspace{0.8ex}\noindent{\underline{\textit{#1}}}}
\newcommand{\ititle}[1]{\vspace{1ex}\noindent\textbf{\textit{#1}}}
\newcommand{\overwrite}[1]{\textcolor{blue}{#1}}
\newcommand{\bfit}[1]{\textbf{\textit{#1}}}
\newcommand{\re}[1]{\noindent{\textbf{\textit{#1}}}}
%%%%%%%%%%%%%%%%%%%%%%%%%%%%%%%%%%%%%
%% DO NOT DELETE!!
%%%%%%%%%%%%%%%%%%%%%%%%%%%%%%%%%%%%%
%\usepackage{tikz}
%\usetikzlibrary{trees}

\usepackage{epsfig}
\usepackage{multirow}
\usepackage{url}

\usepackage[color,matrix,arrow,all]{xy}
%\usepackage[all,cmtip]{xy}

\usepackage{tikz}
\usetikzlibrary{shapes,snakes}
\usetikzlibrary{calc}

\newtheorem{example}{Example}
\newtheorem{theorem}{Theorem}

\newcommand{\add}[1]{\textcolor{blue}{#1}}
\newcommand{\lgl}[1]{\textcolor{blue}{#1}}
\NewDocumentCommand{\cc}{ mO{} }{\textcolor{red}{\textsuperscript{\textit{CC}}\textsf{\textbf{\small[#1]}}}}

\newcommand{\addnew}[1]{\textcolor{red}{#1}}


\NewDocumentCommand{\wjy}{ mO{} }{\textcolor{violet}{\textsuperscript{\textit{WJY}}\textsf{\textbf{\small[#1]}}}}

\NewDocumentCommand{\nan}{ mO{} }{\textcolor{blue}{\textsuperscript{\textit{Nan}}\textsf{\textbf{\small[#1]}}}}

\sloppy
\newcommand{\rtable}[1]{\ensuremath{\mathsf{#1}}}
\newcommand{\ratt}[1]{\ensuremath{\mathit{#1}}}
\newcommand{\stt}[1]{\texttt{\small{#1}}}
\newcommand{\at}[1]{\protect\ensuremath{\mathsf{#1}}\xspace}
\newcommand{\myhrule}{\rule[.5pt]{\hsize}{.5pt}}
\newcommand{\oneurl}[1]{\texttt{#1}}
\newcommand{\tabstrut}{\rule{0pt}{4pt}\vspace{-0.1in}}
\newcommand{\stab}{\vspace{1.2ex}\noindent}
\newcommand{\sstab}{\rule{0pt}{8pt}\\[-2.2ex]}
\newcommand{\vs}{\vspace{1ex}}
\newcommand{\exa}[2]{{\tt\begin{tabbing}\hspace{#1}\=\+\kill #2\end{tabbing}}}
\newcommand{\ra}{\rightarrow}
\newcommand{\match}{\rightleftharpoons}



\newcommand{\la}{\leftarrow}
\newcommand{\bi}{\begin{itemize}}
	\newcommand{\ei}{\end{itemize}}
\newcommand{\mat}[2]{{\begin{tabbing}\hspace{#1}\=\+\kill #2\end{tabbing}}}
\newcommand{\be}{\begin{enumerate}}
	\newcommand{\ee}{\end{enumerate}}
\newcommand{\beqn}{\begin{eqnarray*}}
	\newcommand{\eeqn}{\end{eqnarray*}}

\newcommand{\ie}{$i.e.$,\xspace}
\newcommand{\eg}{$e.g.$,\xspace}
\newcommand{\wrt}{w.r.t.\xspace}
\newcommand{\aka}{a.k.a.\xspace}
\newcommand{\kwlog}{\emph{w.l.o.g.}\xspace}


\makeatletter
\newcommand\figcaption{\def\@captype{figure}\caption}
\newcommand\tabcaption{\def\@captype{table}\caption}
\makeatother

\newcommand{\reminder}[1]{ {\mbox{$<==$}} [[[ \bluefont{ \bf #1 } ]]] {\mbox{$==>$}}}

\definecolor{shadecolor}{RGB}{220,220,220}
\newcommand{\mybox}[1]{\vspace{1.5ex}\par\noindent\colorbox{shadecolor}
	{\parbox{\dimexpr\columnwidth-2\fboxsep\relax}{#1}}\vspace{1ex}}


\tikzstyle{mybox} = [draw=black, fill=black!5, thick,
rectangle, rounded corners, inner sep=0pt, inner ysep=2pt]
\tikzstyle{fancytitle} =[fill=black, text=white]


%#############################################################
\newcommand{\sys}{\texttt{StreamLake}\xspace}
\newcommand{\brain}{\texttt{LakeBrain}\xspace}
\newcommand{\hdfs}{\texttt{HDFS}\xspace}
\newcommand{\kafka}{\texttt{Kafka}\xspace}






%#############################################################











\begin{document}
\title{Separation Is for Better Reunion: Data Lake Storage at Huawei}

%%
%% The "author" command and its associated commands are used to define the authors and their affiliations.
\iffalse
\author{Ben Trovato}
\affiliation{%
  \institution{Institute for Clarity in Documentation}
  \streetaddress{P.O. Box 1212}
  \city{Dublin}
  \state{Ireland}
  \postcode{43017-6221}
}
\email{trovato@corporation.com}
\fi



%%
%% The abstract is a short summary of the work to be presented in the
%% article.
\begin{abstract}
Huawei collaborates with some  Chinese large business companies to store and process exabytes of nationwide operational data in data lake storage to provide business insights.  
 Specifically, our customers will ask to store  and process massive  log message data to support their real-time and  decision making applications. Thus, we need compute and storage components in the analytic platform  to process and store these data cost-efficiently. 


%This process involves handling nation-scale streams and batch data processing with massive static server resources, which requires compute and storage components in the analytic platform  to process and store massive data cost-efficiently. 


 %However, existing solutions are sub-optimal due to \cc{inevitable data isolation and copies in compute-engine-oriented designs.}
 

 To meet these user requirements, we have designed a data lake storage system, \sys, which introduces a novel  design to serve log message streaming and batch data processing  in distributed storage, with high scalability, efficiency, reliability and low cost. 
 Specifically, we introduce a stream object as a storage abstraction for streaming data to achieve the storage-disaggregation architecture with high scalability and reliability. Moreover, we utilize the erasure coding and tiring storage to save the storage cost, and furthermore, the  stream object can be automatically converted to table object such that cost-effective stream and batch data processing can be achieved. For tabular data, we support the lakehouse functionality  to support ACID via the table object, with a metadata cache to improve the efficiency of data access between compute and storage engine.  Also, we design a \brain optimizer at the storage side to optimize the query performance and resource utilization under the storage-disaggregation architecture.
 %Data intensive operations such as message streaming, query operator pushdown, transaction and query time travel are managed inside the centralized disaggregated storage cluster to minimize data copies and shorten time windows in analytic pipelines.
 % Storage features such as multi-level caches and erasure coding along with algorithms like reinforcement learning and probabilistic network are also applied in \sys to further optimize query time and resource usage. 
  Finally, we have also deployed  \sys in China Mobile, the world's largest mobile network operator  to serve over 20\texttt{PB} production data, and the results demonstrate improvements of 30\% to $4\times$ in terms of query performance and over 37\% in terms of cost saving.
\end{abstract}

%Huawei collaborates with some of China's largest business companies to manage and analyze vast amounts of nationwide operational data in data lake storage. This process involves handling nation-scale streams of updates and static server resources, which necessitates the use of sub-systems within the analytic platform to efficiently utilize and share enormous amounts of data. However, existing solutions are suboptimal due to the isolation and duplication of data in compute-engine-oriented designs.

%To address this problem, Huawei has developed an experimental data lake storage system called StreamLake. StreamLake introduces a unique design that supports log message streaming and ETL processing acceleration in distributed storage. To minimize data copies and shorten time windows in analytic pipelines, StreamLake manages data-intensive operations such as message streaming, query operator pushdown, transaction, and query time travel inside a centralized disaggregated storage cluster.

%Furthermore, StreamLake implements storage features such as multi-level caches and erasure coding, as well as algorithms like reinforcement learning and probabilistic networks, to further optimize query time and resource usage. Huawei has tested this framework using production data from China Mobile, the world's largest mobile network operator, and the results demonstrate significant improvements in terms of performance, server resource gains, and query time optimization, ranging from 30% to 4x and over 39%, respectively.

\maketitle

%%% do not modify the following VLDB block %%
%%% VLDB block start %%%
\iffalse
\pagestyle{\vldbpagestyle}
\begingroup\small\noindent\raggedright\textbf{PVLDB Reference Format:}\\
\vldbauthors. \vldbtitle. PVLDB, \vldbvolume(\vldbissue): \vldbpages, \vldbyear.\\
\href{https://doi.org/\vldbdoi}{doi:\vldbdoi}
\endgroup
\begingroup
\renewcommand\thefootnote{}\footnote{\noindent
This work is licensed under the Creative Commons BY-NC-ND 4.0 International License. Visit \url{https://creativecommons.org/licenses/by-nc-nd/4.0/} to view a copy of this license. For any use beyond those covered by this license, obtain permission by emailing \href{mailto:info@vldb.org}{info@vldb.org}. Copyright is held by the owner/author(s). Publication rights licensed to the VLDB Endowment. \\
\raggedright Proceedings of the VLDB Endowment, Vol. \vldbvolume, No. \vldbissue\ %
ISSN 2150-8097. \\
\href{https://doi.org/\vldbdoi}{doi:\vldbdoi} \\
}\addtocounter{footnote}{-1}\endgroup
\fi
%%% VLDB block end %%%

%%% do not modify the following VLDB block %%
%%% VLDB block start %%%

%%% VLDB block end %%%


%!TEX root = ../main.tex
\section{Introduction} 
\label{sec:intro}

%As Internet of Things (IoT) and 5G communication technologies are widely commercialized, massive data are collected, stored and analyzed. The traditional architecture of data infrastructure at data centers has been challenged by cloud-native designs. Compute and storage resources are pooled to be able to serve massive structured and unstructured data in an elastic and cost-efficient manner.  Analytical systems such as data warehouses and big data platforms have also evolved from siloed construction to disaggregated storage and compute architecture with horizontal integration of resource pools. Thanks to its 10x better price, availability and persistence compared to traditional storage formats, data lake storage [4] becomes a de facto cost-friendly storage layer in this architecture, storing massive online and offline data captured for large scale data analysis. Enterprise data engineers and analysts can easily use data lake storage implementation such as AWS S3~\cite{} or Huawei OceanStor Pacific~\cite{} as an affordable and reliable centralized pool of storage to save full data. On top of it, they build data preparation and analytics pipelines to assist business decision making, serve customers and meet compliance requirements. 

As the Internet of Things (IoT) and 5G communication technologies become increasingly prevalent, massive amounts of data are being collected, stored, and analyzed.
 The traditional architecture of data infrastructure  has been challenged by cloud-native designs, where compute and storage resources are pooled to serve massive structured and unstructured data in an elastic and cost-efficient manner.
  Analytical systems such as data warehouses and big data platforms have also evolved from siloed~\footnote{a term?} constructions to disaggregated storage and compute architectures. For example, data lake storage ($e.g.,$ AWS S3~\cite{}, Huawei OceanStor Pacific~\cite{}), with its 10$\times$ better price, availability, and persistence compared to traditional storage formats, has been very popular for storing massive various data, so as to support large-scale data analysis.


However, as large enterprises further digitalize their business, the data to be stored and analyzed explode. Over the past several years, we have collaborated closely with over 200 enterprise customers from 16 different industries to better understand their big data processing requirements. Our analysis of key statistics has revealed the following insights:

\noindent \underline{\textit{Petabytes of data.}} Nearly half of our customers (49\%) have processed data ranging from one terabyte to 10 petabytes (PB). A significant percentage (29\%) handle more than 10 PB, while 8\% manage over 100 PB of data.


\noindent \underline{\textit{Log data.}} A large majority (81\%) of our customers primarily work with log message data.

\noindent \underline{\textit{Stream and batch processing.}} Both stream and batch processing play a critical role in big data processing. 69\% of  customers actively use batch processing, and 65\% use stream processing. Nearly 40\%  care about both. 
Also, when processing data through data pipelines, in many cases, customers  have to continuously  update the datasets.

\noindent \underline{\textit{Data retention.}} In practice, 43\% customers are required to store data between 1 and 5 years. 22\% store between 5 and 10 years and 27\% store at least 10 years, according to regulations and practices in different industries.


To satisfy the above users' requirements, we aim to design a data lake storage system to support stream and batch data co-processing with high efficiency, persistence, scalability and low Total Cost Ownership (TCO). To this end, the system has several significant aspects  to be considered. (1) As users always face petabytes of log streaming data, it is challenging to store the data persistently at low cost, while keeping high scalability and processing efficiency. For example, as streaming data needs real-time processing,  typical system like Kafka uses local file system as the storage, which lacks of scalibility capacity because the computation and storage are tightly coupled. Also, in practice, given the same data, over which users may conduct stream or batch processing for different applications, thus storing two copies for different processes is costly.
 %流的存储很贵,因为要实时处理,所以计算存储强耦合,不能各自scale。扩的时候一起扩,会贵。 
 %分叉了不好办,改了一个影响另一个
% If one considers to store as stream data, and convert it to batch when batch processing is necessary, it will be time-consuming to load the data to the compute engine and conduct the conversion.	
% 数据有多个备份(multiple copy) 更新自己备份 别人不aware 导致数据不一致 delta lake video
(2) In a data analysis pipeline, there are likely to be multiple copies when data goes through the pipeline. If these copies are updates individually, data  will be corrupted because   they are not aware of the updates.   
 Hence, it is significant to support atomic writes to achieve high quality data.
  \cc{does the metadata read latency has relation with the copies?}
%元数据读取时间长
(3) In data lake storage, compute-and-storage \cc{disaggregated} architecture  is applied for scalability,  and thus it is challenging to perform an end-to-end optimization like in a database. Hence, it is critical to consider how to incorporate an optimizer in the storage engine, so as to optimize the query performance and resource utilization.

% 









%The management costs rise rapidly and become a heavy burden. For instance, 4.8 petabytes fresh data flow into data lakes daily for storage and analysis in China Mobile, the world's largest mobile network operator, and the existing architectures of analytic systems are inapt to scale and process petabytes of data gracefully. Hence, expensive additional operations have to be introduced to support analytic needs. Similarly, we have observed the same situation in other large cooperate customers which Huawei closely works with. A large amount of resources and costs have to be added as the analytic system scales. Although storage increase is inevitable as data grow, a large portion of the cost increases in the analytic systems is because compute engines in data pipelines fail to share data management properly. Hence, massive server resources are wasted in data transfers, re-computing and re-storing intermediate states across engines even though these engines may use same data inputs.


%As large enterprises continue to digitize their businesses, the amount of data to be stored and analyzed is growing exponentially. This results in significant management costs that can become a heavy burden. For example, China Mobile, the world's largest mobile network operator, receives 4.8 petabytes of fresh data every day for storage and analysis. However, existing analytic systems are not equipped to handle petabytes of data gracefully. This means that expensive additional operations must be introduced to support analytic needs. This same situation is observed in other large corporate customers that Huawei closely works with, where a significant amount of resources and costs must be added as the analytic system scales. While storage increases are inevitable as data grows, a significant portion of the cost increases in analytic systems is due to compute engines in data pipelines failing to share data management properly. As a result, massive server resources are wasted in data transfers, re-computing, and re-storing intermediate states across engines, even though these engines may use the same data inputs.


To address these issues, we deploy our \sys system with its novel design to serve enterprise-level  \cc{massive message stream ingestion and data pipeline co-processing.}


First, in terms of the streaming storage,
%存储层 分发层的分离,stream object,读写流消息。
 we introduce the \textit{stream object}  to provide efficient stream storage and access. It designs the read/write interfaces to support real-time streaming by \cc{XXX (for read), XXX (for write)}. 
To achieve cost-effective stream and batch co-processing, we also design the \textit{table object} that can be automatically converted  to the \textit{stream object}, and vice versa. In this way, data can be maintained for just one copy rather than storing for two copies separately, and thus the storage cost is reduced. 
Existing works~\cc{\cite{}}, but \cc{XX}.
%自动分级 自动分成冷存储,转成表格式




Second, in terms of supporting updates, Lakehouse system~\cite{} can address this by achieving concurrent read and write in an ACID manner. We also implement the lakehouse functionality in \sys to supports ACID for \cc{both stream and batch data.} Particularly, we design a global write cache that ~\cc{combines small I/O access}, so as to~\cc{XX}.
%元数据是大量小文件读写 

Third, for storage-side optimization, we build an intelligent data lake optimizer \brain that focuses on optimizing the data layout in the storage.%resource utilization + query performance
 To be specific, a reinforcement learning based automatic compaction module is designed to decide whether to compact small files considering the system state, so as to \cc{improve the block utilization while keeping the system running smoothly.} Besides, a predicate-aware partitioning model is utilized to judiciously distribute data to storage blocks to reduce the number of blocks to be visited, so as to improve the query efficiency.  Existing works~\cc{\cite{}},

Overall, our \sys has the following characteristics.


\noindent \underline{\textit{High processing efficiency.}} 


\noindent \underline{\textit{High storage scalability.}}


\noindent \underline{\textit{Low TCO.}} 


\noindent \underline{\textit{High reliability.}}

\noindent \textbf{Use case.} To satisfy the user requirements by achieving the above goals, we build a storage system \sys that \cc{XXX} and deploy it in China Mobile data lakes with production data, \cc{resulting in significant optimization of ?resource utilization?.}  China Mobile manages one of the largest data analytic platforms in China.
Over 4.8 petabytes \cc{per day?} of fresh data flow from business branches and edge devices scattered across over 30 provinces to several centralized data centers. The fresh data first lands on a collection and exchange platform where data exchanges across data centers. Then it is loaded into the analytic platform. Data warehouse and big data engines run billions of jobs  over the data to provide location services, network logging analysis and many other applications to serve users.
As the platform grew to the exabyte scale, \cc{resource utilization} became increasingly skewed, with average CPU, memory, and storage utilizations at \cc{26\%, 41\%, and 66\%} respectively.




To overcome this, we deployed StreamLake in a China Mobile data center with 20 petabytes of production data, replacing the existing analytic architecture with a disaggregated-storage architecture powered by Huawei OceanStor Pacific with  \sys framework.\cc{Moderate changes are applied to connect the analytic engines to StreamLake.}

\cc{The evaluation shows a significant improvement of resource utilization.}
\sys runs the same number of analytic jobs with 39\% less servers, due to the high utilization of \cc{data and server resources} in \sys. Besides, \sys also introduces  benefits in term of performance and service flexibility. For instance, some batch queries can speed up to 4 times when the query operator pushdown and the \brain are enabled. 
For message streaming, originally, they had to maintain 300 more \kafka servers, and the expansion of partitions and nodes posed a big challenge to the China Mobile IT team. With the stream storage in StreamLake, the team no longer needs to \cc{manually} manage the Kafka servers. In addition, minimum data migration is required to scale the system, and thus maintenance costs are thus greatly reduced. 

\cc{which factors are shown in the exp figure??}


\cc{The structural figure is necessary??}

 
 
 
 
 
 
 \iffalse

To address this common problem of big data platforms in enterprise production environments, researchers and engineers in our data management community have evolved existing compute engines or designed new system components to converge data pipelines and enhance data re-usage. For example, first, unified engines~\cite{} for stream and batch processing as well as lakehouse technologies~\cite{} with ACID-compliant transactions are developed. These approaches are closely tied to a specific compute engine, optimizing an end-to-end process with a point view. However, in order to cover various business analytical needs, enterprise production data pipelines need close collaboration between multiple analytic tools. Hence, rather optimizing for a single step or module, it requires us to take an end-to-end view to make effective collaborative optimizations in real world. Second, in the modern cloud-native storage-disaggregated architecture, network overhead is expensive if we maintain a shared state layer in the compute cluster which requires frequent and massive data accesses to the storage. Third, separating data management from the storage layer misses the opportunities to apply advanced storage technologies~\cite{} which are critical to process massive data cost-effectively. Finally, data layout management strategies such as compaction and partitioning are key to ensure overall storage utilization and query performance. Existing approaches~\cite{} normally apply static or manual methods to compact small files or manage data partitions. This is sub-optimal compared to dynamic self-learning algorithms in production with complex data pipelines and massive data. We argue that it is reasonable to dedicate the data management component to the storage side and intelligently optimize it on top of the data lake persistent storage to enable shortest dataflows and to maximize the potential of data re-usage across engines in a cost-efficient and collaborative manner.


In this paper, we present a data lake storage framework, StreamLake, with its novel design to support massive message stream ingestion and data pipeline co-processing. This framework provides community compatible API for message stream processing, allowing inflow messages to bypass the compute layer and to be injected to the data lake storage cluster directly. It also supports ACID-compliant transactions, query time-travel, and query operator pushdowns to minimize data transfer and maximize data sharing. A new data lake optimizer is introduced where reinforcement learning and probabilistic network algorithms are applied to data layout management to optimize query time and server resource usage. All these features are built on top of the Huawei OceanStor Pacific storage and the system applies advanced capabilities in the Huawei storage such as in-cluster RDMA network, tiered storage, erasure coding and instant snapshot~\cite{} to achieve superior system resource utilization with reasonable costs. With this centralized stateful storage, analytic engine instances can be more elastic and hence server resources can be highly utilized. We have experimented this framework in a China Mobile data lake with 20 petabytes of data. The system has demonstrated significant server usage gains. We conclude our contributions as follows:

We propose a novel design of a data lake storage co-processing framework to unify log message streaming, processing and querying acceleration in disaggregated data lake storage, improving both the resource utilization and the speed of data processing pipelines in scale.


We design a scalable message stream storage with separated control and data planes. It offers ecosystem-compatible messaging system APIs and is highly scalable and reliable, being able to directly consume billions of records with cost-efficient tiered persistent storages. 


We extend the message streaming service to support lakehouse data formats for query concurrent processing. It provides lakehouse tabular abstraction to computing engines to concurrently access millions of records as ACID-compliant tables, improving the granularity of the data usage significantly.


We introduce a new data lake optimizer in storage. It applies reinforcement learning and probabilistic network algorithms with storage characteristics and query history to optimize the data layout dynamically, optimizing both the query time and the storage block utilization.

We conduct a use case study with the China mobile IT team to evaluate the StreamLake implementation. Compared to the current system, the experiments show that the new design brings in 39\% to 4x resource usage and performance gains.

\fi

%The rest of this paper is organized as follows. Section 2-6 detail the motivation, design and implementation of key components such like the stream storage, the lakehouse framework and the data lake optimizer. Section 7 discusses the experiments. Section 8 compares related work. Conclusions are drawn in Section 9.

%%!TEX root = ../main.tex
\section{Motivation} 
\label{sec:motivation}



%!TEX root = ../main.tex
\section{Architecture} 
\label{sec:archi}

%To satisfy enterprise customers' needs in the next generation big data solutions, we have designed and implemented \sys, a data lake storage system based on the Huawei OceanStor Pacific storage. The system aims at optimizing the end-to-end processing of massive log messages in big data pipelines. Figure 1 shows the architecture of \sys which consists of store, data service and access three layers from a high-level perspective.

In order to meet the demands of enterprise customers for next-generation big data solutions,  \sys  aims at optimizing the end-to-end processing of massive log messages in big data pipelines. At a high level, \sys is composed of three layers: storage, data service, and data access, as depicted in Figure~\ref{fig:archi}.



 
\begin{figure}[!t]
	\centering
	\includegraphics[scale=0.25]{figures/mobile}
	\vspace{-2em}
	\caption{China Mobile Use Case.}
	\label{fig:mobile}
	\vspace{-2em}
\end{figure}

\noindent \textbf{Store layer} is responsible for data persistence, which consists of SSD and HDD data storage pools, a high-speed data exchange and interworking bus as well as multiple types of storage semantic abstractions \cc{(including block, file, stream, table, etc).}


(1) The data storage pools comprised of SSD and HDD offer reliable management of stored data. The physical storage space on the disks in the storage cluster is divided into slices, which are then organized as logical units across various \cc{servers and disks} to ensure data redundancy and load balancing. The storage pools also implement storage space features such as garbage collection, data reconstruction, snapshot, clone, \cc{WORM and thin provision, do we need to talk about this?}, etc. 

%The data storage pools comprised of SSD and HDD offer reliable management of stored data. The physical storage space on the disks in the storage cluster is divided into slices, which are then organized as logical units across various servers and disks to ensure data redundancy and load balancing. Additionally, the storage pools incorporate a range of storage space features such as garbage collection, data reconstruction, snapshot, clone, WORM, and thin provisioning.


(2) \cc{The data exchange and interworking bus} offers high-speed data transfer and interworking of different storage abstractions.Its advanced features include support for Remote Direct Memory Access (RDMA), which bypasses the CPU and L1 cache to accelerate data transfer speeds. Additionally, the bus leverages intelligent stripe aggregation, I/O priority scheduling, and \cc{other state-of-the-art technologies} to optimize data transfer and processing.
All nodes are interconnected by the data bus to enable high  Input/Output Operations per Second (IOPS), large bandwidth and low latency data exchanges. Furthermore, the bus supports the interworking of different storage abstractions, allowing for the sharing and  access of a single data piece by different interfaces, which eliminates the need for data migration and significantly saves storage space.

(3) The block, file and \cc{other storage abstraction} implement access interfaces to the underlying storage in different semantics. We introduce two new abstractions, stream object and table object, to manage messaging streams and tabular data efficiently. 
Their implementation will be discussed in Section~\ref{sec:datagen}.


\noindent \textbf{Data service layer} provides a rich set of features to enable efficient data management at enterprise scale. For instance, the tiering service offers static and dynamic data migration and eviction between the SSD and HDD storage pools based on tiering policies, which saves the storage cost a lot. The replication service provides periodical replications to remote sites for backup and recovery. Particularly, to further enhance the capabilities of the layer, we have extended it to include specialized services and optimizations for log message processing operations, which include the \sys  services (Section~\ref{sec:dataeva}) to support \cc{real-time streaming and lakehouse functionality}, and \brain (Section~\ref{sec:lakebrain}) to improve the resource utilization and query efficiency.
 
 \cc{Not specific enough:above, how to involve Elastic Serverless Function Engine? }

% These services and optimizations are elaborated in section 5 and 6.
% The elastic serverless function engine is a component that we introduce to support near data processing of the StreamLake services. Its design is discussed in section 5.3. 

%To further enhance the capabilities of the Data Service Layer, we have extended it to include specialized services and optimizations for log message processing operations. These include the StreamLake services and LakeBrain optimization, which are elaborated in Sections 5 and 6 of our report.

%To support near-data processing of the StreamLake services, we have introduced a new component called the Elastic Serverless Function Engine. Its design and functionality are discussed in detail in Section 5.3.


\noindent \textbf{Data access layer} implements storage access protocols to handle user requests. It supports a block service via standard iSCSI access, NAS services via NFS and SMB protocols as well as an object service via S3 protocol, etc.
The new StreamLake services utilize the OceanStor distributed Parallel Client (DPC) which is a universal protocol-agnostic client providing shorter but superfast IO path. 
 %The new StreamLake services utilize the OceanStor distributed Parallel Client (DPC), a universal protocol-agnostic client that provides a shorter but super-fast IO path.
The Access Layer also plays a crucial role in managing authentication and \cc{ACL} permission control, which ensures that only valid user requests are translated into internal requests for further processing, so as to achieve  the security and integrity.
 
 

 
 
 
 
 \begin{figure}[!t]
 	\centering
 	\includegraphics[scale=0.35]{figures/archi}
 	\vspace{-1em}
 	\caption{\sys Storage Architecture.}
 	\label{fig:archi}
 	\vspace{-2em}
 \end{figure}
 
 
 
%!TEX root = ../main.tex
\section{STREAM AND TABLE STORAGE OBJECT} 
~\label{sec:datagen}

In this section, we introduce the stream object and table object, purpose-built storage abstractions designed for efficient storage and access of stream and table data in the storage layer.


\subsection{Stream Object}~\label{subsec:streamobject}

The stream object is a storage abstraction in the store layer that efficiently supports key-value message streaming at scale. It stores a partition\footnote{-- one partition or partitions} of key-value pairs for continuous message streams, organized as collections\footnote{-- a collection or collections} of data slices. Each slice can contain up to 256 records as depicted in Figure~\ref{fig:write}. Incoming message records are appended  to a specific slice in a stream object based on its topic, key, and offset.

\noindent \textbf{Stream objects operations.} The stream object operates similarly to the block and file storage abstractions, providing read and write functionality for stream storage. Figure~\ref{fig:streamobject} outlines key operations supported by the stream object, including creating and destroying a stream object with  functions \texttt{CreateServerStreamObject} (line 1-3) and \texttt{DestroyServerStreamObject} (line 4-5) respectively. The \texttt{$^*$option} field (line 2) sets storage configurations, such as data redundancy methods (replicate or erasure code) and I/O quotas, so as to ensure enterprise-level reliability and performance. The assigned \texttt{objectId} (line 3) serves as a unique identifier for operating the stream object. The \texttt{AppendServerStreamObject} function appends incoming records\footnote{-- an incoming record?}  to the stream object and returns the starting offset of the appended records. The \texttt{ReadServerStreamObject} function reads the stream object starting from a specified offset, with control conditions such as the length of the read specified in the \texttt{readCtrl} field. 
Since the message service is designed to support real-time streaming, it is configured to return all subsequent messages\footnote{*why?} unless \add{specified or reaching quota limits.}\footnote{*The context of this sentence is not clear.} \texttt{IO\_CONTENT\_S} (line 8 and 14) is a data structure that provides non-blocking I/O by using buffers to enhance the performance of both writing and reading operations.


\begin{figure}[!t]
	\centering
	\hspace{2.5em}
	\includegraphics[scale=0.35]{figures/streamobject}
	\vspace{-1em}
	\caption{Stream Object Operations.}
	\label{fig:streamobject}
	\vspace{-1em}
\end{figure}



\noindent \textbf{Write stream messages.} We discuss how to write messages into \sys and endure enterprise-level load-balanced and redundant persistence for the stream objects, which is achieved on the basis of SSD and HDD storage pools.  As shown in Figure~\ref{fig:write}, the messages are first   assigned to stream object slices based on topics, keys, and offsets (Figure~\ref{fig:write}-a,b,c).\footnote{@ how dies it work?} Then, a distributed hash table is leveraged to ensure even data distribution for load-balance storage (Figure~\ref{fig:write}-d). Specifically, data slices will be distributed evenly to 4096 logical shards, each of which has the storage space managed by persistence logs (PLog, Figure~\ref{fig:write}-e)\footnote{--Plog 4096}. Each PLog unit is a collection of persistence services in OceanStor~\cite{} that controls a fixed amount of storage space on multiple disks and provides 128 MB of addresses per shard. When a message is received, the PLog unit replicates it to multiple disks for redundancy (Figure~\ref{fig:write}-f). Key-value databases\footnote{--can we name it specifically?} serve as indexes for PLogs for fast record lookup.



\subsection{Table Object}~\label{subsec:tableobject}
We also extend the storage object layer in StreamLake to support table-like\footnote{-- a term?} operations for more effective data storage and management, similar to lakehouses~\cite{}\footnote{* ``similar to lakehouse'' is weird}. The table storage uses an open lakehouse format\footnote{--a term? what does it mean?} with optimizations\footnote{--what? a part in Fig4?} for faster metadata access. The table abstraction is logically defined by a directory of data and metadata files, as shown an example  in Figure~\ref{fig:tableobject}.

\noindent \textbf{\texttt{Data} directory.} Table objects are stored in Parquet files of the \texttt{data} directory. In this example, the table is partitioned based on the date column, so the data objects are separated into different sub-directories by date. Each sub-directory name represents its partition range\footnote{--partition key?}. The data objects in each Parquet file are organized as row-groups\footnote{--a term?} and stored in a columnar format for efficient data analysis. Footers in the Parquet files contain statistics to support data skipping within the file\footnote{--remove within the file?}.

\begin{figure}[htbp]

	\includegraphics[scale=0.3]{figures/write}
	\centering
	\vspace{-1em}
	\caption{Write Message to \sys.}
	\label{fig:write}
	\vspace{-1em}
\end{figure}


\noindent \textbf{\texttt{Metadata} directory}  keeps track of the table schema, file addresses of the table, its partitions and transaction commits \add{etc.}\footnote{*how to correspond to the three levels? and the following illustration order?}, which are organized into three levels: commit, snapshot, and catalog, as shown in Figure~\ref{fig:tableobject}-(b, c, d).

 \noindent \underline{\textit{Commits}} are \texttt{Arvo} files that contain file-level\footnote{--a term?} metadata and statistics such as file paths\footnote{@directory path}, record counts, and value ranges for the data objects. Each data insert, update, and delete operation will generate a new commit file to record changes to the data object files\footnote{--data object files or data object?}.


\noindent \underline{\textit{Snapshots}} are index files that index  valid commit files for a specified time period. These snapshots document commit statistics such as \add{current file and row count, added files and rows, and removed files and rows}\footnote{--messy} as data operation logs. Along with commits, snapshots provide snapshot-level isolation to support optimistic concurrency control. Readers can access the data by reading from the valid commit files, while changes made by a writer will not be visible to readers until they are committed and recorded in a snapshot. This allows for multiple readers and one writer to access the data simultaneously without the need for locks. 

Snapshots also monitor the expiration of all commits, making them essential for supporting time travel. Time travel queries allow data to be viewed as it appeared at a specific time\footnote{@}. By keeping old commits and snapshots, the table object enables the use of a timestamp to look up the corresponding snapshot and commits, so as to access to historical data.

\noindent \underline{\textit{Catalog}}  describes the table object, including the profile data  such as the table ID, directory paths, schema, snapshot descriptions, modification timestamps, etc. The data and metadata files are stored in the table directory\footnote{@}, except for the catalog, which is stored in a distributed key-value engine~\footnote{--can we name it? or cite} optimized for RDMA and SCM to ensure fast metadata access. The data and metadata files are converted to PLogs in the underlying storage for redundant persistence as discussed above.



\begin{figure}[htbp]
	
	\includegraphics[scale=0.3]{figures/tableobject}
	\centering
	\vspace{-1em}
	\caption{File Organization of \sys Table Objects.}
	\label{fig:tableobject}
	\vspace{-1em}
\end{figure}



















%!TEX root = ../main.tex
\section{Streamlake Data Processing} 
\label{sec:dataeva}

In this Section, we present the data processing services  in the data layer. Driven by practical application scenarios discussed in Section~\ref{sec:motivation}, these services provide a comprehensive, enterprise-level data lake storage solution  to efficiently store and process  log messages\footnote{*why emphasize log messages here? what is the connection with the following sentence?} at scale. The StreamLake services encompass a stream storage system for message streaming (Section~\ref{subsec:stream}), lakehouse-format read/write capabilities for efficient tabular data processing (Section~\ref{subsec:lakehouse}), and support for query operator computation pushdown(Section~\ref{subsec:pushdown}).


\subsection{Message Streaming}~\label{subsec:stream}


\subsection{Lakehouse Read and Write}~\label{subsec:lakehouse}


\subsection{Query Operator Computation Push Down}~\label{subsec:pushdown}
%!TEX root = ../main.tex
\section{Lakebrain optimization} 
\label{sec:lakebrain}

Optimizing large-scale data management and query processing is critical in data warehouse and big data systems, as noted in numerous studies~\cite{}. However, optimizing the end-to-end performance and resource consumption of a complex storage-disaggregated analytics platform with multiple compute engines is a challenging task, due to the lack of information about the compute and storage cluster environments as well as queries executed by other engines. Even if all this data were available, the optimization process would still be difficult due to the large number of interdependent variables and the vast search space~\cite{}.

To address this challenge, we present~\sys, a novel data lake storage optimizer that complements end-to-end data pipeline optimization. Unlike query engine optimizers, which focus on join ordering and cardinality estimation~\cite{}, \sys aims to optimize data usage during query execution, which is key to improving both query performance and storage resource utilization in a storage-disaggregated design. For instance, in a streaming application scenario, data ingestion and transactions often result in numerous small files, leading to low query performance on merge-on-read (MOR) tables. LakeBrain can use compaction to combine these small files into fewer, larger ones, improving inter-cluster storage and network usage as well as query performance.

LakeBrain's design is kept simple for ease of extension and support for different applications. It consists of three components: a statistics collector, the core optimization logic, and an executor. The statistics collector gathers system configurations, environment variables, and workload history, while the core optimization logic employs heuristic rules, probabilistic models, and machine learning algorithms to suggest the best strategy candidates. The executor then deploys the chosen strategy, with its effects being collected as feedback by the statistics collector for future optimization.

To demonstrate the value of a data lake storage optimizer, we have developed two LakeBrain applications: auto compaction and predicate-aware fine-grained partitioning. These use cases will be explained in detail in the following section.

\subsection{Automatic Compaction}

File compaction aims to find the optimal strategy for compacting files that result in improved query execution time or increased block utilization in storage. To achieve this goal, the optimization process employs two algorithms: particle swarm optimization (PSO) and reinforcement learning (RL).

PSO is used to search for the global optimum, a population-based method that doesn't require assumptions about the relationship between tunable parameters and query performance. The goal is to obtain an approximately optimal solution within a limited time. RL, on the other hand, finds a more sophisticated policy based on the states of the data lake environment.

The optimization process involves considering a set of discrete compaction configurations within the action space. Since the state space is continuous, a function approximation method is preferred. The stability of the training process is critical due to the high degree of variability in query performance in a distributed environment, so proximal policy optimization (PPO) is applied.

A deep neural network (DNN) is used to approximate the policy and the value function, with a shared feature backbone network that covers both global and local characteristics of the states. The output from the feature network is processed by two fully connected networks to compute the policy output and the action value. The actor and critic are alternatively updated after collecting new trajectories using the latest policy during training.

Once a desired result is obtained, the numerical output of the compaction strategy is translated into actionable operations by the data lake connector for a specific data lake engine, facilitating the optimization process.


\subsection{Predicate-aware Fine-grained Partitioning}

Optimizing data partitioning involves assigning records to storage blocks in the most efficient manner possible, thereby reducing the number of blocks accessed during queries. Our partitioning approach is based on the query-tree framework [43], and utilizes a sum-product network (SPN) probabilistic model [19, 26, 31] to model the distribution of the data in LakeBrain. This is done in order to ensure fast inference speed and avoid repeatedly scanning the datasets.

The query-tree framework creates a tree-based partitioning strategy using pushdown predicates. Each leaf node represents a partition, and its column ranges are derived from the pushdown predicates used to split its parent nodes. By using probabilistic models to characterize the dataset, we can identify the most suitable partitioning policy. The probabilistic model-based cardinality estimation, as demonstrated in Figure 11, is used to estimate the number of records in each partition, instead of scanning the original data, thereby saving a significant amount of time. This greatly improves the speed of the partitioning optimization algorithm, making it suitable for large-scale systems.

Additionally, probabilistic models allow us to represent a sequence of datasets with a series of probabilistic models that have a fixed structure but varying parameters. This is achieved by representing a sequence of datasets as a series of multi-dimensional vectors, each representing the learnable variables in the probabilistic model with a fixed length, i.e. a time series. We can then use time series prediction methods to predict future probabilistic models, and use these predictions to estimate the number of records in a partition during partitioning optimization.

To implement the optimized data layout, we introduce a partitioning mechanism that saves data in fine-grained partitions based on the partitioning strategy. Additionally, we have implemented an evaluator that skips irrelevant partitions by checking the overlap between pushdown predicates and the column ranges in each partition. For numerical columns, the range can be represented as lower and upper bounds, which are well-handled by many data formats. For categorical columns, we either record its range or its complement using "IN" or "NOT IN" predicates. The effectiveness of this predicate-aware partitioning approach is evaluated in section 7.2, where the test results show exceptional performance.

%!TEX root = ../main.tex
\section{Experiment} 
\label{sec:exp}


\subsection{Settings}

\subsection{Evaluation of Message Streaming}

\subsection{Evaluation of LakeBrain}

\subsection{Evaluation of Query Pushdown}


\subsection{China Mobile Use Case}
%!TEX root = ../main.tex
\section{Related Work} 
\label{sec:related}


In this section, we will discuss relevant open-source projects and systems related to \sys $w.r.t.$ \cc{streaming platforms}, lakehouse data management, query computation pushdown and automatic database tuning.


\cc{Data lake storage system} 

%\noindent\textbf{Query computation pushdown.} NetApp~\cite{} supports Hadoop to use its storage devices through NFS-based connector docking, through S3A docking to its object storage, and through SAS/iSCSI/FC building native \hdfs on its block/lun devices. AWS EMRFS~\cite{} is an enhancement introduced to address the inconsistency of object storage in metadata operations, with official information showing that it has made computation pushdown related optimizations for the engine. Alibaba EMR is based on object storage, and the JindoFS~\cite{} solves the performance problem of object storage by introducing local data caching. These solutions improve data access to the persistent storage in computation pushdown while StreamLake offers built-in computation pushdown operations directly. 

\noindent\textbf{Streaming platforms.} Kafka, Pulsar and Pravega~\cite{} are widely-used open-source streaming platforms in industry. Unlike \sys, which builds its messaging service on top of the stream object and PLogs, and integrates its stream storage with a lakehouse framework, these solutions are file-based and require manual connections to compute engines and external storage, such as \hdfs~\cite{} or \texttt{S3}~\cite{}, for downstream processing or cost-friendly archiving. This increases both the complexity and cost of data pipeline management.



\noindent\textbf{Lakehouse.} Iceberg, Hudi and Delta Lake~\cite{} are popular  lakehouse data management framework, which rely on statistic file or object storage. 
Massive data transmission between the storage and the compute engines are inevitable in many scenarios. \cc{Not coherent!!!}
 \sys builds the lakehouse framework on top of the table object and PLogs, leveraging the enterprise-level data redundancy, high performance cache and query computation pushdown to provide reliable and high speed concurrent lakehouse reads/writes. 









%NetApp supports Hadoop by providing several connectors that enable Hadoop to use its storage devices. These connectors include an NFS-based connector for docking with NetApp's storage devices, an S3A connector for docking with NetApp's object storage, and native building of HDFS on NetApp's block/LUN devices through SAS/iSCSI/FC.

%AWS EMRFS is an enhancement introduced to address the inconsistency of object storage in metadata operations. Official information shows that AWS EMRFS has made computation pushdown related optimizations for the engine, which improve data access to persistent storage.

%Alibaba EMR is based on object storage, and it uses JindoFS to solve the performance problem of object storage by introducing local data caching. This solution improves data access to persistent storage in computation pushdown scenarios.

%StreamLake offers built-in computation pushdown operations directly, making it easy for users to take advantage of this feature without having to configure complex connectors or caching mechanisms.

%Overall, these solutions provide various ways to improve data access to persistent storage in computation pushdown scenarios, whether through native building of HDFS on block/LUN devices, improved metadata operations, or data caching.

\noindent\textbf{Automatic database tuning .} 
Recently, AI is widely-used inside the database system to improve the performance. OtterTune [40] is a classic ML-based framework, recommending knob configuration using Gaussian process (GP). To address the limitation of traditional ML-based approaches, RL has been adopted in CDBTune [44]. Investigated in [41] shows the impact of the performance variation in production environments, indicating that GP tends to converge faster but is frequently trapped in local optima, whereas RL or deep learning (DL) generally needs a longer training process and achieves better performance.  [37] is the first approach that tries to maximize data skipping for a partitioning using pushdown predicates with a bottom-up approach. QDTree [43] proposed a greedy algorithm and a reinforcement learning based algorithm to solve the data skipping maximization problem to solve the suboptimal limitation. However, these algorithms need to quantify the performance of each candidate partitioning. In addition, the partitioning layout is sub-optimal when new data comes, as it is optimized based on existing data.
%!TEX root = ../main.tex
\section{Conclusion} 
\label{sec:con}

In this paper, we build a data lake storage system \sys that facilitates stream and batch data processing with high elasticity, reliability, scalability, efficiency. We adopt the compute-and-storage disaggregated architecture to achieve  elasticity and reliability via the stream object and table object. We implement the lakehouse functionality to support ACID for tabular data with a global write cache to achieve acceleration.  Moreover, we design \brain at the storage side so as to achieve query and resource optimization in the storage  disaggregated architecture. We have deployed \sys in Chine Mobile to storage and process 20PB production data with high performance and low cost.



\newpage


%\begin{acks}
% This work was supported by the [...] Research Fund of [...] (Number [...]). Additional funding was provided by [...] and [...]. We also thank [...] for contributing [...].
%\end{acks}

%\clearpage

\normalem
\bibliographystyle{ACM-Reference-Format}
\bibliography{bib/ref}

\end{document}
\endinput
